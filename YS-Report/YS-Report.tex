\documentclass[11pt, a4paper]{article}
\usepackage[utf8]{inputenc}
\usepackage[margin=1.2 in]{geometry}
\usepackage{graphicx}
\usepackage[T1]{fontenc}
\usepackage{charter}
\usepackage{amsmath}
\usepackage{setspace}
\usepackage{subcaption}
\usepackage{listings}
\graphicspath{{Assets/}}

\title{CSCU9YE Assignment Report}
\author{Student no: 2519302}
\date{November 17\textsuperscript{th} 2018}
\pagenumbering{roman}
\begin{document}
\begin{titlepage}
\maketitle
\thispagestyle{empty}
\end{titlepage}
\doublespacing
\setstretch{2}
\tableofcontents
\newpage
\singlespacing

\section{Introduction}

BIG ASS TODO

\section{Pre-Processing}

This section will aim to cover the different aspects of pre-processing employed in this project. Within the datasets used in this project, we can see that the general steps one would need to take are as follows:\\
\begin{itemize}
\item Lemmatization
\item Removal of stop words
\item Removal punctuation
\end{itemize} 

\subsection{Lemmatization}

\textbf{First version:} Lemmatization is a technique employed to remove inflection and/or derive the base form of a word in the dataset. As an example, the words, \emph{cleaning}, \emph{cleaner}, \emph{cleanliness} can all be reduced to its base word, \emph{clean}.\\\\
Such mapping then allows us to find all relevant documents using a specific word. \\\\
\textbf{Second version:} Lemmatization is a form of text normalization where the aim is to remove inflection and/or derive the base word from a family of words in the dataset. As an example:
\begin{center}
\begin{tabular}{c c c}
\hline
\textbf{Word} & & \textbf{Lemma}\\
\hline
\text{Cleaning} & \(\rightarrow\) & \text{Clean}\\
\hline
\text{Cleaner}  & \(\rightarrow\) & \text{Clean}\\
\hline
\text{Cleanliness} & \(\rightarrow\) & \text{Clean}\\
\hline
\end{tabular}
\end{center}

\subsection{Stop Words}

//todo

\subsection{Punctuation}

//todo
%\section{Machine Learning Methods}

%\section{Datasets Used}

%\section{Measurement Metrics}

%\section{Results}


\end{document}
